\documentclass{article}
\usepackage[utf8]{inputenc}
\usepackage{tikz} 
\usepackage[utf8]{enumerate}
\usepackage[T2A]{fontenc}    
\usepackage[utf8x]{inputenc}


\begin{document}

\begin{center}
\textbf{Graph №21\\}


\begin{tikzpicture}[node distance={20mm}, thick, main/.style = {draw, circle}] 
\node[main] (1) {$1$}; 
\node[main] (2) [above right of=1] {$2$}; 
\node[main] (3) [below right of=1] {$3$}; 
\node[main] (4) [above right of=3] {$4$}; 
\node[main] (5) [above right of=4] {$5$}; 
\node[main] (6) [below right of=4] {$6$}; 
\node[main] (7) [below right of=5] {$7$}; 
\draw[->] (7) to [out=105,in=90,looseness=1.5] (1);
\draw[->] (5) -- (1); 
\draw[->] (3) -- (1); 
\draw[->] (3) -- (2); 
\draw[->] (6) to [out=105,in=0,looseness=1.5] (2);
\draw[->] (2) -- (4); 
\draw[->] (3) -- (6); 
\draw[->] (5) to [out=250,in=20,looseness=1.2] (3);
\draw[->] (4) -- (6);
\draw[->] (4) -- (7);
\draw[->] (4) -- (5);
\draw[->] (5) -- (7);

\end{tikzpicture} 


\caption{\\Рисунок 1 Ориентированный граф $G=(I,U)$}\vspace{4mm}

\begin{tikzpicture}[node distance={25mm}, thick, main/.style = {draw, circle}] 
\node[main] (4) {$4$}; 
\node[main] (5) [below left of=4] {$5$}; 
\node[main] (7) [below  of=4] {$7$}; 
\node[main] (6) [below right of=4] {$6$}; 
\node[main] (3) [below  of=5] {$3$}; 
\node[main] (1) [below  of=7] {$1$}; 
\node[main] (2) [below  of=6] {$2$}; 
\draw[->] (4) -- (5);
\draw[->] (4) -- (6);
\draw[->] (4) -- (7);
\draw[->] (5) -- (3);
\draw[->] (7) -- (1);
\draw[->] (6) -- (2);

\end{tikzpicture} 
\caption{\\Рисунок 2 Покрывающее дерево графа $G$ (Рисунок 1)}

\begin{tikzpicture}[node distance={25mm}, thick, main/.style = {draw, circle}] 
\node[main] (5) {$5$}; 
\node[main] (3) [below left of=5] {$3$}; 
\node[main] (4) [below right of=5] {$4$}; 

\node[main] (7) [below left of=4] {$7$}; 
\node[main] (6) [below right of=4] {$6$}; 
\node[main] (1) [below  of=7] {$1$}; 
\node[main] (2) [below  of=6] {$2$}; 
\draw[->] (5) -- (4);
\draw[->] (4) -- (6);
\draw[->] (4) -- (7);
\draw[->] (5) -- (3);
\draw[->] (7) -- (1);
\draw[->] (6) -- (2);

\end{tikzpicture} 
\caption{\\Рисунок 3 Корневое дерево с корнем в узле 5}\vspace{4mm}




\textbf{Таблица 1 Структуры для представления корневого дерева (корень - узел 1)\\}\vspace{1mm}
\begin{tabular}{|l|l|l|l|l|l|l|l|}
\hline
$i$             & 1 & 2 & 3 & 4  & 5 & 6 & 7 \\ \hline
$pred{[}i{]}$   & 7 & 6 & 5 & 5  & 0 & 4 & 4 \\ \hline
$depth{[}i{]}$  & 3 & 3 & 1 & 1  & 0 & 2 & 2 \\ \hline
$thread{[}i{]}$ & 6 & 5 & 4 & 7  & 3 & 2 & 1 \\ \hline
$dir{[}i{]}$    & 1 & 1 & 1 & -1 & 0 & 1 & 1 \\ \hline

\end{tabular}
\end{center}
\text{\\Для $\forall i$ в корневом дереве существует дуга $(pred{[}i{]}, i)$. Если в исходном графе (рисунок 1) существует дуга $(pred{[}i{]}, i)$, то $dir{[}i{]} = 1$. В противном случае  $dir{[}i{]} = -1$. Если $i$ является корнем $dir{[}i{]} = 0$. }
\end{document}